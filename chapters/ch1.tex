\documentclass[../main.tex]{subfiles}
\begin{document}
\begin{figure}[bh]
\centering
 
\label{fig:img1}
\caption{None}
\end{figure}
A few ideas to keep in the introduction (STILL MESSY)

1. We have an excessive amount of information\\
2. We also have an excessive amount of computational power\\
3. This information can be of use to make conclusions about our world\\
4. Machine Learning takes advantage of these two.\\
5. What sorts of problems can we solve? \\
6. We can learn stuff about our world by building structures and then reasoning about them. \\
7. One such structure is a phylo-acoustic tree, i.e. a tree that shows relationships between animal species in terms of their vocalisations. 
8. If we were able to draw one such tree - what kind of conclusions would we able to draw from it? \\
9. Naturally, we would expect to see "similar birds" closely tied in the tree. Similar birds in this sense would be those that sing similarly. What birds would we expect to sing similarly? Some examples would be those birds that belong to the same genus, or the birds that belong to the same family or order. \\
10. A way of empirically proving this hypothesis would be to find a way to build a phyloacoustic tree, and measure if the underlying clusters make sense in the way described above.\\
11. This process largely covers three steps: find a way of representing relevant information from birdsong (feature extraction). This representation will be useful to calculate similarity between bird species. \\
12. What problems do we have:

a. how do we extract and represent relevant information from birdsong recordings.
b. how do we calculate similarity between bird species.
c. how do we build a phyloacoustic tree from the information above.
d. how do we evaluate a particular phyloacoustic tree.\\
13. In this thesis work I present a way of building and evaluating phyloacoustic trees. 
14. We see that...
15. This piece of work is relevant because...
16. The work is presented in the following way:
- A literature review is given in Chapter 2
- The problem and goals are presented in Chapter 3
- A solution and its implementation are discussed in Chapter 4
- Experimentation results are analysed and discussed in Chapter 5
- Chapter 6 presents a conclusion and a brief overview of future work.
17. Without further ado, we begin. \cite{Awatade2012}
\end{document}