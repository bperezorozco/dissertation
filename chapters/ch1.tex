\documentclass[../main.tex]{subfiles}
\begin{document} \label{chapter_intro}
\section{Background}
Machine Learning (ML) has become ubiquitous. Lying at the intersection between Statistics and Computer Science, its applications cover a broad range of fields: automatic speech recognition, face identification in visual media, sentiment analysis and tumour classification are just a few of the tasks that can be solved by means of ML methods. Although the development of new models and algorithms is mostly a concern of Statistics and Computer Science, several fields of application have contributed to broaden its horizons by demanding to solve increasingly complex tasks. 
\par If algorithms and statistical models are the tools ML works with, then data can be regarded as its raw material. Simultaneous to the development of ML, vast amounts of data have become available in recent years. This is due crucially to the increasing provision of devices and services that humans use to capture data. Microphones have allowed the continuous recording of natural environments, surveillance cameras have greatly increased the amount of video available, and the development of web services, such as social networks, have encouraged everyday users to upload their own content to the web. 
\par These phenomena are directly related to \emph{citizen science}, which is a term describing the users that do not exercise science as a profession, but still collaborate to gather data by means of their own devices. Citizen science has been a catalyst in the exponential growth in the amount of data in the world in recent years, which, as of 2015, is estimated to be 7,910 exabytes (1 exabyte = $2^{18}$ bytes), 50 times more than it was only ten years ago \cite{TheEconomistOnline2011}.
\par This increased availability of data, alongside the increasingly widespread use of computational methods across other fields of study, has contributed to make ML techniques a standard in data analysis. In particular, scientists are often concerned by finding hidden structure in their data in order to gain further insight. ML techniques that aim to solve this task form part of a branch called \emph{unsupervised learning}. 
\par Consider, for instance, the bewildering number of bird species common to Europe, whose range, diversity and biodiversity are not well understood. Projects such as Xeno-Canto \cite{Xeno-cantoFoundation2015} and the Animal Sound Archive \cite{AnimalSoundArchive2015} have benefited from citizen science in order to offer large bird vocalisation datasets for these species. Using this data, scientists can contribute to the quantification of environments across Europe.
\par Quantification of environments is important because it allows us to analyse phenomena in a formal manner. In particular, interaction between species, interaction between of individuals, migration patterns and environmental diversity are some examples of phenomena that can be analysed by means of the datasets described above.

\section{Problem definition and justification}
Now consider the problem of modelling the similarity between two bird species by means of their vocalisations. Given a set of bird species and a similarity metric, it is possible to visualise this set as a relational structure, and draw conclusions from it. In particular, we could ask if this graph does have a structure. Proving this hypothesis would have strong implications in the biological sense, since we would be showing that there is an informative relation between the vocalisations emitted by different bird species.
\par In this dissertation, we aim to build a phylo-acoustic tree, which is a hierarchical relational structure with the properties described above. Adding a hierarchical property to it will enable us to discover the structure of data at different levels of granularity, and hence to draw finer conclusions about it. 
\par There are two perspectives from which building this relational structure provides insight. One is the zoological perspective, in which the relational structure unveils interactions between bird species that can be further explained by biological theories. For example, a tree containing subtrees with mostly species from the same genus (first word in the scientific name) indicates that birdsong might be linked to bird evolution. A second example would be two species that are also clustered in a subtree, but rather than belonging to the same genus, they inhabit the same area. This would suggest that bird vocalisation is closely related to geographical location of bird species.
\par Nonetheless, making a formal evaluation of biological aspects is outside the scope of this project. We instead focus on the insight provided by a ML perspective of the problem. Building a phylo-acoustic tree from bird vocalisations opens a discussion in the following points:
\begin{itemize}
\item As will be reviewed in section \ref{birdsong_review}, birdsong shares many traits with vocal human sounds. This suggests that techniques used to analyse human speech could be extrapolated to birdsong and produce results of the same quality, which is interesting from a \emph{feature extraction} (see section \ref{features_review}) point of view. By the same reasoning, results of techniques applied to birdsong could be later extrapolated to analyse human speech, hence also benefiting the field of Automatic Speech Recognition.
\item Although looking for specific phylo-genetic or geographical relations between species is not part of the project, we are concerned by proving that there does exist a structure in our data.
\item Enabling a metric over bird species vocalisations requires proposing a model that is abstract enough to overcome the difficulties of comparing birdsong feature sequences (such as different length or small numerical perturbations). Tackling this challenge requires finding a way of quantifying birdsong uncertainty.
\item Similarity metrics research is of great interest to ML, and having to define one for the models described above that characterise bird vocalisations can pose to be a challenge depending on the complexity of the model.
\end{itemize}

\section{Project goals}
Having described our problem and justified it, we now set out the goals of this dissertation:
\begin{enumerate}
\item To build a phylo-acoustic tree of bird species by means of bird vocalisations only. 
\item To define a method to build relational structures from bird vocalisations. This pipeline should be generic enough so as to permit ``filling in the gaps'' with one among a range of techniques.
\item To provide a broad literature review so as to give several options to implement the proposed method. Even though not all the techniques will be implemented in this work, this state of the art directly sets out future work by describing other techniques that can be used to build relational structures.
\item To infer optimal feature representations for distinguishing bird species. In order to achieve this, we will benefit from acquiring an understanding of how birdsong is produced.
\item To propose a statistical model that symbolically characterises a species' vocalisation. This encourages characterising each bird species as an abstract model that quantifies the uncertainty and variability in birdsong production across different individuals of the same species.
\item To find a similarity metric between bird species. This is equivalent to finding a metric between the abstract models described in the previous goal.
\end{enumerate}

\section{Dissertation outline and work carried out}
This dissertation is organised as follows:
\par In chapter \ref{chapter_soa}, we define a general pipeline to produce relational structures and a review on the biological aspects of birdsong production. Moreover, we give an extensive literature review on feature extraction techniques and relational structure algorithms.
\par Chapter \ref{chapter_formants} gives an introduction to Digital Signal Processing that will allow us to introduce the framework of Linear Prediction of Speech. More importantly, the latter enables us to introduce formant trajectories, our chosen feature extraction technique.
\par Chapter \ref{chapter_hmms} aims to describe statistical models to symbolically learn formant trajectories. In particular, we discuss Kernel Density Estimation, Gaussian Mixture Models and Variational Bayes Hidden Markov Models. We then present similarity metrics between these models and give a further description of Agglomerative Hierarchical Clustering, a method used to build arborescent relational structures.
\par Then, chapter \ref{chapter_account} gives an account of the implementation and experiments carried out in this project.
\par Finally, chapter \ref{chapter_conclusion} provides a summary of the contributions offered in this dissertation and describes future research.

\end{document}