\documentclass[../main.tex]{subfiles}
\begin{document} \label{chapter_results}
The main goal of this chapter is to describe what we are evaluating and how we are doing it. Measures of evaluation:

1. How many formants to use? Metric: show the variance of each F1, F2, F3 and the mean bandwidth. More relevant formants have smaller variance and mean bandwidth.
2. Species related in real life are closer together under these measures than any random species : for each species with more than 1 member, show the cardinality of the real tree and the mean of some randomly generated trees
3. Compare the number of clusters over time between random and "real" trees. The former will look like an upwards concave curve, the latter will look like the tail of a Gaussian.

\section{Choosing the number of formants} \label{section_choosing}
\end{document}