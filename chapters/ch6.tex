\documentclass[../main.tex]{subfiles}
\begin{document} \label{chapter_conclusion}
In this final chapter, we offer a more profound discussion on the results presented in chapter \ref{chapter_account}. This will lead us to highlight the main contributions of this dissertation and to discuss future work.

\section{Discussion of results}
In chapter \ref{chapter_account}, we presented the implementation details for a MATLAB prototype that builds phylo-acoustic trees automatically and then showed 6 examples of them. As described in chapters \ref{chapter_intro} and \ref{chapter_soa}, it is difficult to measure what a ``good tree'' is, since there is no real ground truth to it - birdsong is not an evolutionary trait, hence there is no guarantee that a phylo-genetic tree can be fully recovered by analysing vocalisations. 
\par On the contrary, a wide variety of phenomena affect how birdsong evolves, e.g. adaptation of migrating bird flocks, environmental pressures and mating advantage all affect how birdsong changes over time. All these factors suggest that phylo-acoustic trees could be used to highlight relations that do exist among bird species, and that they could be interesting to analyse from a zoological perspective.
\par Nevertheless, some relations are straightforward to observe in phylo-acoustic trees, even for people outside of the field. For example, consider the \emph{Phylloscopus} genus. In all six phylo-acoustic trees in appendix \ref{app_phylo}, species belonging to this genus tend to cluster together, suggesting the traits they share can be identified even by means of different techniques.
\par Another example is related to the genera \emph{Phylloscopus}, \emph{Sylvia} and \emph{Acrocephalus}. Species belonging to these genera are often clustered together - sometimes even before being grouped with other species of their same genus. This comes to no surprise when we discover that all three genera are what we call warblers in standard English, suggesting that the tree does show some connection to phylo-genetic relations. Further discussion regarding this kind of observations is outside the scope of this project.
\par However, we can also draw conclusions from a mathematical perspective. As shown in section \ref{subsection_null}, our phylo-acoustic trees do exhibit a community structure. This key aspect allows us to confirm that trees are not being randomly generated and that the relations exhibited in them do exist in out data. 

\section{Summary of contributions}
In this section, we aim to highlight the contributions introduced in this dissertation. Our main contribution is a method to create relational structures from bird vocalisations. To the author's knowledge, building phylo-acoustic trees from birdsong has not been done before in the literature. Instead, most of the work on computational analysis of birdsong has focused on classification of bird species. Consequently, building relational structures becomes interesting from several viewpoints: in Zoology, it opens many new (and potentially unexpected) questions regarding the interaction of bird species; on the other hand, in Computer Science and Mathematics, it broadens their horizon of application and encourages the research of problems that are still open to discuss (for example, Hidden Markov Model comparison).
\par Additionally, a general pipeline to build such structures has been defined in chapter \ref{chapter_soa}. This general pipeline serves as a template to be followed in order to build relational structures, and is accompanied by a broad literature review which describes several techniques that can ``fill in the gaps'' of each stage of the pipeline. This literature review serves as a baseline to set forth future work, as will be described in section \ref{section_future}.
\par Moreover, we are setting a precedent in the usage of formant trajectory extraction from birdsong. Although it is clear that birdsong production follows the source-filter model of speech (which leads to the definition of the Linear Prediction of Speech framework reviewed in chapter \ref{chapter_formants}), there is little evidence in the literature that formants have been used before as features for birdsong. The results in our work highlight the possibility of extending the usage of formant trajectories as features for other tasks, such as classification. 
\par In addition, we also offered an empirical analysis on the structure of formants over time. This contribution, although not being a formal proof, did show exhaustive evidence about why the trajectory of the first formant was more relevant than any other. It is important to note that this can serve as a baseline of future work on the information theoretic aspects of formant trajectories, i.e. how does information content vary as formants appear in higher frequencies? Although there is a broad literature that addresses this for human speech, this is not the case for birdsong, hence justifying further research in this field of application.
\par One further contribution we made was the usage of statistical models to summarise feature vectors. To the author's knowledge, this approach has been seldom used as an intermediate step in Machine Learning, since the standard approach consists in using feature vectors directly as input to learning algorithms. This key step is one of the most relevant contributions in this work, given that it directly addresses structural issues of feature vectors. In particular, not only does it address the variability of vector dimensionality across the dataset, but it also encourages \emph{symbolic} learning of feature vectors. 
\par We allowed for symbolic learning by using probability distributions and Hidden Markov Models as intermediate structures that summarise the sequential structure of feature vectors. As a result, the similarity between bird species becomes the similarity between their statistical models, rather than vectors that are sensitive to stretching transformations.
\par Finally, we also discussed similarity metrics between probability distributions and between Hidden Markov Models. Although the former consisted mostly in a summary review of metrics often used in the literature, we went a step further for the latter by defining similarity metrics of our own: on one hand, the Approximate Bayesian Inference framework allowed us to incorporate time sequence information to the training of the HMMs' emission models, thus one of our metrics consisted in using a probability distribution metric between two emission models and consider the resultant to be the HMMs' similarity. On the other hand, we also defined a metric considering the HMMs' transition models, which consisted in taking the Symmetric KL Divergence between pairs of states, and weighting each component using the \emph{occupancy} of the state. We defined the occupancy to be a probability distribution that measures how much each state is used in average.
\par Most of these contributions have opened new questions which are relevant to this research. This future work is discussed in section \ref{section_future}.

\section{Future work} \label{section_future}
Some of the contributions described in this chapter have opened new questions which could be researched in the future. We now discuss these questions.
\par Formant analysis for human vowels has suggested that the first three formants are enough to characterise them. However, it is not clear how this result extends to birdsong. As discussed in chapter \ref{chapter_account}, the first formant is the most structured one according to our evidence, but this does not imply that it gives sufficient information to characterise birdsong completely. How many formants should we use then? The additional constraint of guaranteeing that this number either holds for all bird species, or that there is a method to find it for every bird species, adds yet another level of complexity to this question.
\par Another concept we discussed was HMM similarity. There is some literature on how to perform this computation, and we also proposed two different approaches. However, there is no established, \emph{by-default} method to do this. Moreover, most of the research on this topic only considers a subset of the parameters of the HMM to perform calculations, and assumes the rest of the parameters are only indirectly involved by affecting the training of those parameters that are taken into account. Research on this topic concerns statistical learning as a subject, and hence has a broader scope of application.
\par Finally, different choices for the general pipeline stages have yet to be explored. In particular, the broad literature review presented in chapter \ref{chapter_soa} introduced several alternatives to generate phylo-acoustic trees, and even other types of relational structures. Further exploring these options might lead to more results of potential interest to both, Computer Scientists and Zoologists.

\section{Closing comments}
In this project, we described a method to automatically build phylo-acoustic trees from birdsong. This enables users to visualise interactions between bird species, and give a meaningful interpretation to them, which could be justified by a real-world phylo-genetic, environmental or behavioural relation. In order to describe this method, we made an extensive review of techniques from Digital Signal Processing, Statistics and Machine Learning, and combined it with contributions where needed. These included using statistical models as intermediate steps to encourage symbolic learning of patterns, defining similarity metrics and providing empirical evidence of arguments for feature selection. The resulting phylo-acoustic trees were shown to have a real community structure and, at the same time, to display interactions that do take place in real life (e.g. clustering species that belong to the same genus).
\par In conclusion, we have presented a clear example of how Machine Learning has impacted other fields of study (in this case, Zoology), but at the same time broadening its own horizons and discovering further questions to be answered. Along with the explosion of information available to analyse, this versatility has increasingly made Machine Learning one of tools of choice to analyse data. Furthermore, this positive feedback process between Machine Learning and other fields has also enabled humans to further their understanding their world. An example of this is the phylo-acoustic trees we have built, where we unveiled interactions between real bird species that may have been carried from a long time ago. Having Computer Science and Mathematics as its weapons of choice, there is little doubt that Machine Learning will continue to be a revolution in our age.

\end{document}