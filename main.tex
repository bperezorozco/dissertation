\documentclass[pdftex,12pt,a4paper]{report}
\usepackage[utf8]{inputenc}
\usepackage[english]{babel}
\usepackage{natbib}
\usepackage{graphicx}
\usepackage{subfiles}
\usepackage{amsmath}
\usepackage{amsthm}
\usepackage{amssymb}
\usepackage{amsfonts}
\usepackage{mathtools}
\usepackage{graphicx}
\usepackage{titlesec}
\graphicspath{{images/}{../images/}}
\usepackage{geometry}
 \geometry{
 a4paper,
 total={210mm,297mm},
 left=30mm,
 right=30mm,
 top=30mm,
 bottom=35mm,
 }
 
 \newcommand*{\justifyheading}{\raggedleft}
\titleformat{\chapter}[display]
  {\normalfont\huge\bfseries\justifyheading}{\chaptertitlename\ \thechapter}
  {20pt}{\Huge}
\newcommand{\defeq}{\vcentcolon=}
\newcommand{\eqdef}{=\vcentcolon}
\newcommand*{\V}[1]{\mathbf{#1}}%
\newcommand{\norm}[1]{\left\lVert#1\right\rVert}
\newcommand{\justif}[2]{&{#1}&\text{#2}}
\newcommand{\qedwhite}{\hfill \ensuremath{\Box}}
\renewcommand{\qed}{\hfill\blacksquare}
\renewcommand{\baselinestretch}{2}

\DeclareMathOperator*{\argmax}{arg\,max}
 
\title{Learning relational structures from birdsong}
\author{Bernardo Pérez Orozco\\Balliol College\\University of Oxford}
\date{ September 1st 2015 }
\begin{document}
 \begin{titlepage} 
\begin{center}

% Upper part of the page. The '~' is needed because \\ % only works if a paragraph has started. \includegraphics[width=0.15\textwidth]{./logo}~\\[1cm]

\textsc{\LARGE Learning relational structures from birdsong}\\[2cm]
\includegraphics[width=0.35\textwidth]{./images/logo.png}\\[2cm]
% Author and supervisor 

{
\Large 
\textbf{Bernardo Pérez Orozco}\\
\textsl{Balliol College\\
University of Oxford\\}
\vspace{15mm}
\textbf{Supervised by:}\\
\textsl{Prof. Stephen Roberts \& Prof. Thomas Melham}
}
\vfill

% Bottom of the page 
{\Large 
\textsl{Dissertation submitted for the degree of}\\
\textsc{Master of Science in Computer Science}\\
\textsl{September 1st 2015}
}

\end{center} 
\end{titlepage}

\chapter{Introduction}
\subfile{chapters/ch1.tex}
 
\chapter{Literature review}

\subfile{chapters/ch2.tex}

\chapter{Feature extraction and signal analysis}
 
\subfile{chapters/ch3.tex}
 
\chapter{Similarity measures and relational structures}

\subfile{chapters/ch4.tex}

\chapter{Implementation}

\subfile{chapters/ch5.tex}

\chapter{Empirical results}

\subfile{chapters/ch6.tex}

\chapter{Conclusion}

\subfile{chapters/ch7.tex}

\bibliographystyle{unsrt}
\bibliography{bib}

%Formant implementation\\
%http://uk.mathworks.com/help/signal/ug/formant-estimation-with-lpc-coefficients.html\\
%http://www.phon.ucl.ac.uk/courses/spsci/matlab/lect10.html\\
%http://dea.brunel.ac.uk/cmsp/home_yan_qin/intro/FmtEstimation.htm\\
%http://practicalcryptography.com/miscellaneous/machine-learning/guide-mel-frequency-cepstral-coefficients-mfccs/\\
\end{document}